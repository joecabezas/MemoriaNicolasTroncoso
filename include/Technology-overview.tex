\chapter[Technology Overview]{Technology Overview}
\label{ch:technology-overview}
This chapter glides over the current
code generator technologies
that are relevant in the ACS framework context.
The main scope of this thesis
is the generation of backend code,
for that reason only tools suited for this purpose
are discussed.

Program transformation has applications and uses
in many areas of software engineering,
including compilation,
optimization~\cite{grune02:_compiler_design},
refactoring,
program synthesis,
software renovation,
and reverse engineering~\cite{muller00:_rever_engin, RE-WEB}.
The aim of program transformation is to increase programmer productivity
by automating programing tasks,
thus enabling programing at a higher-level of abstraction,
and increasing maintainability and re-usability~\cite{visser01:_asurvey}.

Program transformation may be divided roughly into the areas of
translation and rephrasing.
A \emph{translation} involves transforming a program from a source language
into a ``different'' target language.
Examples of such transformations are compilers such as
GCC~\cite{GCC-WEB},
TAO IDL~\cite{TAO-WEB},
\mbox{OmniOrb} IDL~\cite{OMNIORB-WEB},
jacOrb IDL~\cite{JACORB-WEB}.
Others are model to code transformations,
like the
ACS Error definition system~\cite{jeram02:_ACS_error_system}.

\emph{Rephrasings} are transformations that take an input program
and produce an output in the same language.
Rephrasing tries to say the same thing using different words.
The main scenarios where this is used are optimization,
refactoring,
and renovation.

\section{Generic Program Transformation (Code Generation)}
\label{sec:translation}

The main focus of the survey in this section
is the assessment of translation technologies
for use in the ALMA project.
This case study is aimed at finding proper tools and methods
to transform high level program definitions
into a high level language implementation
as is done by TAO, OmniOrb and JacOrb when transforming IDL representations.
This survey covers generic tools and special purpose tools.
The objective is to review
different technologies.

\subsection{IDL Transformations}
\label{sec:IDL-transformations}
Interface definition language,
or IDL for short,
is a specification language used to describe a software component's interface.
This language is part of the CORBA specification~\cite{CORBA-WEB, OMG-IDL, CORBA-OMG02},
which makes it a commonly used standard,
in particular it is widely used in ACS since ACS is CORBA-based.
IDL is the language used to describe the
interfaces that client objects call and object implementations provide. An interface
definition written in IDL completely defines the interface and fully specifies
each operation’s parameters. An IDL interface provides the information needed
to develop clients that use the interface’s operations.
This description may then be used
to generate language specific implementations of such interfaces,
making it possible to call objects written in one language
from clients written in another in a transparent manner.
The following are implementation examples of IDL transformations,
in particular the ones in common use in ACS.
\begin{description}
\item[The ACE ORB (TAO)~\cite{TAO-WEB}:]
  A CORBA C++ Object Request Broker (ORB).
  This tool offers an IDL compiler that creates a C++ representation
  of the interface.
\item[omniORB~\cite{OMNIORB-WEB}:]
  A robust high performance CORBA ORB that
  offers an IDL compiler for C++ and Python.
\item[jacOrb~\cite{JACORB-WEB}:]
  A CORBA Java ORB that
  provides an IDL compiler for the Java language.
\end{description}
These are perfect examples of a program translator (code generator).

\subsection{Code Generation in VLT Software}
\label{sec:code-generation:VLT}
Very Large Telescope (VLT) Software~\cite{filipi99:_vlt_software}
is a common proprietary middleware platform
which provides common services such as messaging, persistence, error handling and logging.
It also provides several utility programs, some with graphical user interface, to make development and debugging
of VLT applications easier.
To ease the development of VLT applications a dedicated framework is used.
The Workstation Software Framework (WSF)~\cite{andolfato09:_wsf}
is a state machine model driven development
designed to generate applications based on ESO VLT software.
Where state machine models are used to generate executables.
The framework provides code generation and supports hierarchical state machines, generating readily usable code.

\subsection{ALMA Control Software: Hardware Device Code Generator}
\label{sec:control-generator}
ALMA Control software uses a model driven approach
to generate code that controls the underlying hardware~%
  \cite{farris06:_generating_software_modules}.
Information from the hardware ICD (Interface Control Document)
is captured in the form of spreadsheets (using Microsoft Excel 2003)
that contains detailed information about the device,
its monitor and control points,
and its data monitor points that are to be permanently archived.
The spreadsheets are saved as XML documents~%
  \cite{farris07:_device_driver_code_gener_framew}.
Such a XML document is later fed to a template based code generator
that produces code that runs seamlessly in the ACS framework.
The generator used is OAW (Open Architecture Ware)~%
  \cite{OAW-WEB}, a modular generator framework implemented in Java.
It supports parsing of arbitrary models,
and a language family to check and transform models
as well as to generate code based on them.

An important factor to consider in this code generator framework
is the automatic generation of a simulation environment~%
  \cite{mora08:_towar_gener_hardw_devic_simul}.
This enables developers to test their code as soon as possible
getting a minimum assurance that the new code works.

The shortcoming of this code generator
is that it is specific for ALMA Control Software,
and it is detached from the ACS software release cycle.
This means that extra modifications are required by the CONTROL
software team when a new ACS delivery includes changes in the framework.
Notwithstanding, the Control Software generator is in production
(it is considered to be delivered and complete)
and no new design enhancements are foreseen in the near future.
An enhancement beyond the scope of this undergraduate thesis
would be to integrate the control generator with the new
generator proposed by the author.

\section{Code Generation in ACS}
\label{sec:code-generation:ACS}
ALMA Common Software as a framework supports code generation to some extent.
CORBA stubs (see~\ref{sec:IDL-transformations})
and error definitions (see~\ref{sec:ACS-error-system})
for all languages supported by ACS are automatically generated.
Also,
ACS provides a state machine code generator
which to some extent makes ACS a generated code framework.
Having all the previous generation facilities,
the core programing is done in the software components.
Nevertheless,
writing software components is still a repetitive task,
the overhead is the same for all components~%
  \cite{grega05:_acs_comp_cont_fram_tut}.
To circumvent the overhead of repetitive tasks some
ACS code generators have been proposed, like
the ACSgenerator described in~\ref{sec:acs-generator}.

\subsection{ACS Error System Code Generator}
\label{sec:ACS-error-system}
The ACS Error System~\cite{jeram02:_ACS_error_system}
is based on XML (Extensible Markup Language)
and XSLT (Extensible Stylesheet Language Transformations)~%
 \cite{XSLT-WEB} code generation.
Types and their corresponding codes are defined in XML files
in an ACS language independent way.
Which will in turn be converted to the specific
Java,
C++ and
Python implementation.
For each error definition type,
a unique XML file must be written by the developer.
This file is used by the XSLT transformation to produce the code.
The generated code contains a set of completion and exception helper classes,
one for each error code.
There is a separate XSLT for each programming language
and one specifically for IDL.
Currently all code generation is done automatically
during system build
by using a target in the ACS \lstinline[language=sh]!Makefile!.
This code generator is considered to be complete,
at this time it is being used in production
in ALMA software.

\subsection{ACSGenerator}
\label{sec:acs-generator}
ACSgenerator is a community code contribution to the ACS project.
It is based on bdsGenerator~\cite{ACS-CONT}
which was created to use ACS sofwate
with the Hexapod Telescope~\cite{HPT-WEB}.
The bdsGenerator system is a C++ hardcoded template based code generator
that parses IDL files and creates
a C++ ACS component module.
ACSgenerator further extends the bdsGenerator work
providing very advanced code generation.
At this time ACSgenerator
only creates C++
component modules making it
complementary to the work in this thesis.
The work proposed in this thesis aims
at solving some limitations of ACSgenerator,
such as
one interface per module,
and simple template extension.
The expirience of this generator should be used
when extending the ACS generator to the C++ implementation
in an effort to merge the functionality of both.

\section{Impact of Code Generation}
\label{sec:code-generation:impact}
During the past decade
there has been an
upsurge in the use of code generators
in an attempt to deliver enterprise
software to the market as
quickly as possible~\cite{ACG}.
Even though
automated code generation is not a new concept,
it had received
relatively little recognition.

Expertise is an intangible but unquestionably valuable commodity.
People acquire it slowly,
it takes a long time for a novice to become an expert.
Design patterns~\cite{gamma94:_design_patterns}
are a way to capture expert knowledge
and convey it to novice developers.
The drawback with this approach is that a design pattern
is only know-how and not an implementation per se.
Budinsky et al. describe an automatic code generator~%
   \cite{budinsky96:_automatic_code_generation}
which addresses this issue by generating code
using a specified design pattern.
This approach ensures consistency throughout the development.

Support of software evolution and maintenance is at hand
when generating code from a formal definition language.
Such a definition may be appropriately translated
to cope with underlying platform changes
without major developer interaction~\cite{floch95:_automatic_code_generator}.
A framework change is possible without cumbersome refactoring
of all the software built on top of the framework,
since the change can be made in the generator.

Some of the mayor concerns at hand
when working with code generation
is
keeping the generated code
clean and humanly readable.
This is important for this project since
the way defined by ALMA software development
to handle problems in generated code
is to debug
the generated code,
manually
tweaking it until the problem is found and fixed,
and the fix is then introduced into the generation process.
This debug process can be extremely complex if the
generated code is not well structured.
Another important pitfall of the code generation
is that if a bug is introduced in the generated code
this bug will be present all over the generated software.
At the same time, since the generation process is common,
fixing a bug
fixes it in all the generated software.
On the other hand,
if the generator itself is opaque,
there will be a tendency to fix the generated code instead of the generator.
These issues should be kept in mind when
working with code generation.

%%% Local Variables:
%%% mode: latex
%%% TeX-master: "../thesis"
%%% End:
