\chapter{Conclusions}
\label{ch:conclusions}
This work has produced a design and a
working implementation of a Model Driven
generator for the ACS framework.
The development process was based on an iterative process with hands on
experience and real examples that made shortcomings and modifications needed
in the design evident.
The shortcomings detected were addressed in the evolution to the actual implementation,
achieving a maturity such that the generated IDL files were
used for the 6th ACS Workshop Basic Track Live Example (see section~%
\ref{sec:external-review}).

The most remarkable features of the generation framework are:
\begin{itemize}
    \item The generated module is ready for compilation and usage as generated.
    No further development is required for initial testing.
    \item A pluggable module design was achieved, in the sense that adding
    an additional implementation language does not disrupt, nor cause great
    impact in the generation framework.
    \item The version of OAW used during development is the same as
    the one shipped by ACS-8.1. No porting efforts are required for integration.
\end{itemize}

The work of this undergraduate thesis complies
with the objectives defined in
section~\ref{sec:objectives} as follows:
\begin{enumerate}
\item
  \textbf{Code generation starting from a model
  (and/or text representation).}
  The input for the generation process is an UML model in the
  XMI (UML\,2.0) format.
  \begin{enumerate}
     \item \textbf{Create IDL files. This implies creating a full implementation of the
     UML model as an IDL file so it may be compiled by IDL compilers.}
     A full IDL implementation of the model is generated.
     This includes IDL structs and enumerated types, entities which were not
     considered in the preliminary design, but found to be essential during development.
     \item \textbf{Create base class implementation starting from a class diagram.
     This base classes will be functional as they will implement the
     relevant interfaces of the ACS framework.}
     Basic base class implementation is achieved, these base classes
     also have empty implementation of the complete interface defined by the IDL file.
     \item \textbf{Create CDB schemas. CDB is the Configuration Data Base, this database informs
     the framework about the available software components.
     The database checks the component against an XML schema that defines the software
     component properties.}
     CDB schemas are created where they are needed (ACS Characteristic Component).
     As an addition it was found necessary to create a test deployment CDB.
     This CDB is ready for use, and requires no developer intervention to be used as is.
     Doing this allows the developer to readily test the generated component,
     without spending time on the time consuming labor of defining a CDB.
     \item \textbf{Define the type mapping between the UML model representation
     and the specific language implementation.}
     This is internal to the code generator templates.
     Mappings to convert UML types to IDL and Java types
     are defined by extension methods.
  \end{enumerate}
\item
  \textbf{Integrate design patterns to the code during the generation process.}
  By using the code separation pattern the generated code is isolated from
  the developer implemented code. This allows the generator to use and insert
  patterns and abstraction levels without major impact on the hand-written code.
\item
  \textbf{Generate code for one of the ACS supported languages;
  in particular, Java.}
  Full support for Java components and Java Characteristic components was implemented.
\item
  \textbf{Validate the proposed solution through external opinions,
  current experience,
  and prototype implementation.}
  This objective has been achieved as described in chapter~\ref{ch:verification}.
\end{enumerate}

\section{Future Work}
\label{sec:future-work}
This work is an initial prototype
that provides a good base for a full featured
code generator for the ACS framework.
To achieve full featured status,
further work on this generator should implement
the following features and enhancements:
\begin{itemize}
   \item Support for C++ and Python ACS component implementations,
     and perhaps languages to be added in the future.
     Use the experience with other generators,
     like
     the ACSGenerator and the bdsGenerator,
     as a guide for this task.
   \item It should be possible to define the resources
   an ACS component will need during design, this way it would
   be possible to check usage at compile time, and ensure proper
   acquire and release cycle of such resources.
   \item Move common operation logic from manually implemented
   modules into the generated code.
   \item Identify common code chunks or repetitive chunks,
   and move them from the templates into their classes.
   They should be integrated using
   the \texttt{<<uses>>} code pattern.
   \item Flexible package definition, in the sense that it should be
   possible to define in the model
   the way the software modules should be grouped
   into packages, libraries and modules.
   \item Inheritance support between generated and non-generated code.
   This is required to provide support for more complex models.
   \item Provide exceptions support, in the sense that the model and
   generated file should make use of them, and the XML definition
   should be generated so ACS may use the ACS Error Generator to create
   the Exception definitions for all used languages.
   \item Provide an ACS model profile so the model definition does not
   have to rely on stereotypes.
   Currently it uses the type defined in the profile directly,
   that way new components and new characteristic components directly
   extend the ACS component and ACS characteristic component.
   \item Integrate the code generator into the ACS module workflow.
   This should define how code is generated, whether outside the ACS modules
   or inside the same ACS module. How are \lstinline[language=sh]!Makefile!s handled in the later case?
   \item Provide full model checking using an XSD or the OAW check facility.
   The model should be validated by the generator in the sense
   that it should guarantee that it can generate a sensible
   output from it. This is done in the ALMA Control Software,
   which might be used as a starting experience.
    \item Study the feasibility of integrating this work with the
    ALMA control software hardware device code generator
    described in section~\ref{sec:control-generator}.
\end{itemize}

%%% Local Variables:
%%% mode: latex
%%% TeX-master: "../thesis"
%%% End:
