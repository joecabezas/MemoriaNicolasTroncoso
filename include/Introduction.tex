\chapter{Introduction}
\label{ch:introduction}

Chile has huge potential for astronomical observation,
because it is the home of the majority of the most important observatories
in the world,
and the favorite place for further projects planned
to be constructed during the next decade.
With the clearest skies in the world,
Chile is becoming an astronomical power.
Some of the already installed international observatories are
La Silla (NTT, ESO-3.6),
Paranal (VLT),
Cerro Pach\'on (Gemini, SOAR),
Cerro Tololo (Blanco),
and Las Campanas (Magellan).
They are mainly financed by European and North American public organizations.

In addition,
the currently largest astronomical project
is under construction on the Chajnantor plateau
in the Atacama desert:
The Atacama Large Millimeter/submillimeter Array (ALMA)~\cite{ALMA-WEB}.
This will be a radioastronomy observatory,
formed by at least 54 12-meter diameter antennas and
12 7-meter diameter antennas,
which will act together as one big interferometer.
ALMA is a partnership between Europe (ESO),
North America (NRAO), and Japan (NAOJ).
It is expected to be fully operational by late 2012.

As part of the cooperation with the Chilean government,
ALMA is funding scientific development at Chilean Universities
with around half a million dollars annually.
In this context,
the Computer Systems Research Group (CSRG)
at Universidad T\'ecnica Federico Santa Mar\'\i a
is collaborating since 2004 with the development of ALMA software.
This work is currently financed by the ALMA-CONICYT fund,
and is done in close coordination
with the ALMA ACS and Control subsystems development.

Besides the work for the ALMA project,
CSRG is also collaborating with the Chilean astronomy community in other ways.
Some work has been done together with astronomy departments
at some other Universities,
implementing parts of the ALMA software to be used in smaller optical telescopes.

The following description of ALMA software follows~\cite{avarias08:_DDS_vs_NS}.
ALMA software
is built on top of the ALMA Common Software (ACS)~%
 \cite{chiozzi01:_ACS,gianluca06:_ACS_overview}
infrastructure framework.
%This is a CORBA based framework that
%aims to be an extensible platform
%for astronomical and non astronomical projects.
%Having this in mind this document presents a review
%of the current status of code generation,
%it benefits, and its applicability in ACS.
The ACS framework consists of a collection of common
patterns, components, and services. The heart of ACS is an component model
based on Distributed Objects implemented as CORBA~\cite{CORBA-OMG02} objects.
ACS is based on experience accumulated with similar science projects in the
astronomical and particle accelerator contexts, reusing concepts and
technology used previously like VLT Common
Software~\cite{sivera01:_vlt_overview}. Although designed for ALMA, ACS can
and is being
used in other control systems and distributed software projects, since it
has been implemented to be a generic control framework using generic design
patterns and components off the shelf~%
\cite{chiozzi08:_enabling_softw_sharin}.
Through the use of standard constructs and components, non-ACS developers
can easily understand the architecture of software modules. This makes
maintenance affordable even on a very large software project like ALMA.


As part of the ALMA-CONICYT fund collaboration,
CSRG has already contributed to ALMA software in
various aspects, such as a study of real time capabilities~%
  \cite{araya08:_verifiying_ACS_RT_periodic_properties,%
       tobar09:_real_time_time_service_ACS}
, in the area of control software simulation framework~%
  \cite{mora08:_hardw_devic_simul_framew_alma_contr_subsy,%
        mora08:_towar_gener_hardw_devic_simul}
, developed a scheduling prototype algorithm~%
  \cite{saez09:_sched_protot_amateur_teles_contr_acs},
  a generic automatic pointing system~%
  \cite{staig08:_refer_implem_gener_autom_point_system},
  and an amateur telescope control system~%
  \cite{tobar08:_csat_gtcs, tobar08:_CSAT_final_status}
  a hardware end to end example~\cite{valencia07:_H3E} for
  the ACS framework,
  and studied the feasibility of DDS as a replacement for the notification
  channel~\cite{avarias08:_DDS_vs_NS}.

As part of this collaboration,
specifically through project ALMA-CONICYT~\#31060008,
the author presents an undergraduate
thesis to cover some novel aspects which require attention
but have not yet been developed by CSRG nor the ALMA computing group.

\section{Problem Approach}
\label{sec:problem-approach}
Modern software control systems
involve multiple subsystems
and devices.
Projects start with final designs to be implemented,
but during the development of the project
the designs are subject to redefinitions,
redesigns and changes.
As~\cite{farris07:_device_driver_code_gener_framew,%
         farris06:_generating_software_modules}
adequately state,
coding a large codebase requires an
army of programmers.
A better approach is to use model-driven programing,
where the code is generated automatically
by a generator, based on a model.
This directly impacts the error per line of code
(since it takes over many routine, error-prone tasks)
and helps enforcing a single coding style for
a large portion of the code base.
Benefits of such an approach are that
it makes it easier to accommodate model changes,
and
it improves productivity since developers
can focus on business logic
instead of details of implementation.
Besides the above benefits,
improvements in the generator translate directly
into improvements in the whole generated code base.

\section{Objectives}
\label{sec:objectives}
ACS (Alma Common Software) could do with more automatically generated code.
It should be possible to write software components that use the framework
in an easier way than its done today.
Having this in mind is that the author proposes a thesis that accomplishes:
\begin{enumerate}
\item
  Code generation starting from a model
  UML~\cite{UML-WEB}
  (and/or text representation). UML (Unified Modeling Language) is a
  software modeling specification maintained by the Object Management Group (OMG)~%
  \cite{OMG-WEB}
  \begin{enumerate}
     \item Create IDL~\cite{OMG-IDL} files. This implies creating a full implementation of the
     UML model as an IDL file so it may be compiled by IDL compilers.
     \item Create a base class implementation starting from a class diagram.
     Base classes will be functional as they implement the
     relevant interfaces of the ACS framework.
     \item Create CDB XML schemas. CDB is the Configuration Data Base, this database informs
     the framework about the available software components.
     The database checks the component against an XML schema that defines the software
     component properties.
     \item Define the type mapping between the UML model representation
     and the specific language implementation.
  \end{enumerate}
\item
  Integrate design patterns to the code during the generation process.
\item
  Generate code for one of the ACS supported languages;
  in particular, Java.
\item
  Finally,
  validate the proposed solution through external opinions,
  current experience,
  and generating example implementations of a variety of models
  through a prototype implementation.
\end{enumerate}

\section{Document Structure}
\label{sec:documentStructure}
The document is structured as follows:
In chapter~\ref{ch:technology-overview} a review of
available technologies is made,
presenting different code generator
frameworks upon which this thesis is based.
Section~\ref{sec:code-generation:ACS} talks about
the current status of code generation in ACS,
section~\ref{sec:code-generation:impact}
details benefits that this thesis proposes.
Chapter~\ref{ch:architecture} talks about the
architecture of the proposed solution.
Chapter~\ref{ch:verification} details the verification
of the proposed solution.
Section~\ref{sec:test-configuration} sets the
testing framework used for validation.
%Section~\ref{sec:results-comparisons} presents the results
%of the validations.
Finally chapter~\ref{ch:conclusions} shows the conclusions
and final remarks of this undergraduate thesis.

%%% Local Variables:
%%% mode: latex
%%% TeX-master: "../thesis"
%%% End:
