\chapter{Architecture of the Solution}
\label{ch:architecture}
Following the industry standard and a requirement from
the ACS software group,
UML (Unified Modelling Language)~\cite{UML-WEB} was chosen as the model definition language.
As ACS ships with Open Architecture Ware (OAW)~\cite{OAW-WEB}, it was natural to use
this tool to build the ACS code generator
since it avoids introducing new external
software into the ACS framework.

The greatest challenge to achieve a proper code generation for the ACS framework
was to properly translate an UML model into generated code.
The overall architecture proposes to start from
an UML model in XMI 2.0 UML~\cite{OMG-WEB} format,
and use Open Architecture Ware to transform this model into an actual implementation.
The generated implementation consists in
the Java component code,
the corresponding IDL interface,
an XSD Schema file
and
an example configuration database usable for immediate testing.
The generated Java component is a ready component
in the sense that it may be instantiated
by the ACS component without need of extension by the developer.
This provides and easy and first verification that the generated code is working properly.

The general architecture of this code generator
allows the developer to generate ACS
components and
characteristic components.
The former are the simplest case and the later add
additional complexity when generating code
since they require access to the BACI interface.
BACI, the Basic Control Interface
is a general purpose control framework.
It does not specify any specific control system,
but it restrict the space of definable objects
within a specific design and guidelines defined
in the BACI interface~\cite{plesko05:_baci}.
It is not required to have components and characteristic components
in different models.
The ACS framework is container/component based.
Hence ACS components are the basic
functional software element that can be created.
In case the control interface is needed
it is possible to define characteristic components
instead of simple components,
this will give the software implementation
access to BACI.
The ACS code generator
expects that classes identify their
behavior, either as components or as characteristic components,
using stereotypes and thus instructing the
code generation on how to proceed.

The generation flow will take the UML model as input and generate
the IDL interface, Java base code,
and a test CDB (ACS Configuration DataBase).
In the case of a class that is a
characteristic component,
a XSD (Extensible Markup Language Schema Document)
schema is generated as well.

\section{UML Model Specification}
\label{sec:uml-model-specification}
The UML model must comply with some specific stereotypes not specified in the UML standard
but relevant in the ACS framework. These stereotypes are enumerated in table~\ref{tb:stereotypes}.
\begin{table}
\centering
\begin{tabular}{|l|l|}
\hline \textbf{Stereotype} & \textbf{Description} \\
\hline \multicolumn{2}{|l|}{\textbf{Class Stereotype}} \\
\hline \texttt{<<CharacteristicComponent>>} & Indicates if the component is characteristic.\\
\hline \texttt{<<NOGenerated>>} & Indicates that this class must not be generated.\\
\hline \texttt{<<IDLStruct>>} & Indicates that this class must be treated as an IDL struct.\\
\hline \multicolumn{2}{|l|}{\textbf{Type Member Stereotype}} \\
\hline \texttt{<<ReadOnlyProperty>>} & The type uses the read-only BACI interface for properties.\\
\texttt{<<ReadWriteProperty>>} & The type uses the read/write BACI interface for properties.\\
\hline \multicolumn{2}{|l|}{\textbf{Function Member Stereotype}} \\
\hline \texttt{<<Asynchronous>>} & The method is implemented as an asynchronous method.\\
\hline
\end{tabular}
\caption{Stereotypes}
\label{tb:stereotypes}
\end{table}
Additionally,
it should be possible to specify data types which are not defined by the UML specification,
for example the \lstinline[language=IDL]!Sequence! data type. This type is declared and used extensively in the ACS framework.
If this non-standard data type is used in the model it should be used automatically in the generated code without
any special intervention.
Additionally the UML package name will be used by the code generator to name the \lstinline[language=sh]!jar! file, and the directory structure.
An important feature in the generation process is to define some class as
\texttt{<<NOGenerated>>}, this instructs the code generator
not to generate any implementation for it.
This is significant when doing complex extensions,
were it is possible that the developer does not want to generate code for a particular class,
but wants to keep the model complete so the generator may define extending classes
appropriately.

\section{Infrastructure}
\label{sec:infrastructure}
Along with the generated code,
the ACS module infrastructure is created as well.
This infrastructure is needed to properly compile and install and
ACS component,
independent of its implementation language.
This infrastructure is divided in various sub-directories
which contain essential information divided in different categories.
\begin{quote}
  \begin{description}
  \item[idl:] This directory contains the IDL interfaces and the XML
    definitions of the errors.
  \item[include:] This directory contains files with definitions,
    normally C++ header files.
  \item[src:] This directory contains the actual code and the ACS
    \lstinline[language=sh]!Makefile!.
  \item[config:] This directory contains the ACS schema specification
    if one is required.
  \item[doc:] This directory contains any additional documentation.
  \end{description}
\end{quote}
The directory structure is created automatically during the generation process.

\section{Non Generated Code}
\label{sec:non-generated-code}
It is not possible to fully generate the logic within the software component.
To achieve the full functionality of the component the developer must extend the generated code.
This scheme has been shown to work in the ALMA CONTROL software code generation~%
\cite{farris07:_device_driver_code_gener_framew,%
      farris06:_generating_software_modules}
where the generated code is extended by class inheritance.
The same approach is used in the ACS Code Generator,
a minimal working component is automatically generated,
but the internal logic must be added by the developer by means of class extension.

\begin{figure}[htp]
  \begin{center}
  \includegraphics[width=\textwidth]{images/generation}
  \end{center}
  \caption{Code Separation}
  \label{fg:generation}
\end{figure}

The main benefit of this approach is that it isolates
the generated code from the user written code.
The side effect of this is that code may be regenerated
multiple times without affecting user implemented code.
This is known as a code separation pattern as
seen in figure~\ref{fg:generation}.

During the development cycle
the base (generated) classes are created,
then the developer extends these classes
and implements complex functionality within the extensions.
This approach is used with the generated IDL
and generated Java files.
It is also possible to use the same
approach with the generated XSD files.

OAW provides a way to tell the generator which
parts it must generate,
by allowing the developer to delimit certain
regions of code as protected.
The code separation scheme described above was preferred over
region delimitation since the later
adds complexity to the generator
and
to the generated files themselves.

\section{Deployment Configuration}
\label{sec:deployment-generation}
The CDB is the database that holds the deployment configuration for ACS components.
It is not possible to deploy an ACS software module without it.
The code generator presented in this document generates a basic test CDB.
This CDB contains the Manager and components definition.
Every component (characteristic and non-characteristic) defined in the UML model is
inserted into the generated CDB. Thus, once code generation is done, the developer has
a ready to go test CDB with which he may readily instantiate the ACS components.

An additional feature which was specified as a requirement for this code generator
is the creation of a schema file when dealing with characteristic components. So, in addition
of creating the proper CDB files, the generator creates the schema file that validates the
CDB.


\section{Generation Workflow}
\label{sec:generation-workflow}
This section describes the steps taken
in order to convert an UML model into an
ACS system,
see figure~\ref{fg:generalWorkflow} for a general layout
of the steps to accomplish
the end to end generation task.
\begin{figure}[htp]
  \begin{center}
  \includegraphics[width=5cm]{images/general-workflow}
  \end{center}
  \caption{General Generation Workflow}
  \label{fg:generalWorkflow}
% Fixme: Place "IDL tasks", etc in an external column?
\end{figure}

Figure~\ref{fg:processWorkflow} shows the internal steps taken by the code generation.
The targets described in table~\ref{tb:oawtasks} are
tasks taken by the ACS code generator in order to generate
the ACS software module.
\begin{table}[htp]
\centering
\begin{tabular}{|l|l|}
\hline \textbf{Task Name} & \textbf{Description} \\
\hline \multicolumn{2}{|l|}{\textbf{IDL tasks}} \\
\hline \texttt{genidlCommon} & Generate common IDL files. Contains structs and enumerated types.\\
\hline \texttt{genidl} & Generate IDL files for every ACS component.\\
\hline \multicolumn{2}{|l|}{\textbf{Java Tasks}} \\
\hline \texttt{genjava} & Generate Java Helper and Java Base classes.\\
\hline \multicolumn{2}{|l|}{\textbf{ACS module tasks}} \\
\hline \texttt{genschema} & Generate Schemas in case of ACS characteristic component.\\
\hline \texttt{gencdb} & Generate test CDB.\\
\hline \texttt{genmake} & Generate Makefile.\\
\hline
\end{tabular}
\caption{OAW Tasks}
\label{tb:oawtasks}
% Fixme: Place "IDL tasks", etc in an external column?
\end{table}

The tasks described in table~\ref{tb:oawtasks}
show the different OAW targets
which are processed
in order to
generate an ACS software module.
Every task in independent of the
others, so
they may be interchanged, eliminated or
new steps could be added.
\begin{figure}[htp]
  \begin{center}
  \includegraphics[width=10cm]{images/oaw-workflow}
  \end{center}
  \caption{Generation Process Workflow}
  \label{fg:processWorkflow}
\end{figure}


\section{Extension and Modification of the Code Generator}
\label{sec:extension-modification}
As described before it is possible to
extend or modify the templates.
This can be done without big
invasions of the existing templates,
since custom templates may be defined
and their location
configured
in the \lstinline[language=sh]!workflow.oaw! file
under the targets described in table~\ref{tb:oawtasks}.
The same way alternate templates
are configured,
additional targets can
be added
to \lstinline[language=sh]!workflow.oaw!,
since as described before the
code generation targets are independent of each other.
This modularized approach was taken to enable extension and
to keep generation modules small and maintainable.

Extending or designing new templates
requires some expertise and knowledge of
the Xpand2 and Xtend languages,
yet the implemented templates
may be used as a staring point or tutorial
on how Xpand2 works.
OAW documentation clear, good, abundant and freely available,
and OAW has recently been merged with the
Eclipse Modeling Project~\cite{EMF-WEB},
widening its community support.

\section{Final Proposal}
\label{sec:final-proposal}
This section describes the final proposal for the ACS code generator.
This proposal takes into account
all the previous discussion,
further validation may be seen in chapter~\ref{ch:verification}.

\subsection{OAW Generation Process Specifics}
\label{sec:oaw-specifics}
OAW (Open Architecture Ware)~%
\cite{OAW-WEB} is a modular generator framework implemented in Java.
It supports parsing of arbitrary models,
and a language family to check and transform models
as well as generate code based on them.
It provides the following facilities:
Xtext, Check, Xpand2 and Xtend.
The architecture of the ACS code generator
is based on the later two.
Xpand2 provides a template base generation,
and Xtend provided a set of helper functions
to keep the templates clean.

The tasks that OAW executes are defined in the
workflow configuration (\lstinline[language=sh]!workflow.oaw! file).
This is an XML file which configures how
the model is parsed and which templates
should be applied to the model.

\begin{itemize}
  \item{\textbf{Parsing and navigating the model.}}
  OAW provides the flexibility to define
  in house model representations,
  as seen in the case of ALMA CONTROL Software~\cite{farris06:_generating_software_modules},
  or use factory shipped model parsers.
  In the specific case of the ACS code generator,
  since UML models are used,
  it is possible to use the UML model parser provided
  by OAW.
  The model representation is easily navigated
  since it is presented as an hierarchical structure,
  in the form of a series of containers related to other,
  where it is possible to ask for
  attributes, contained elements, or parent elements.
  The ACS code generator is designed to
  understand class diagrams.
  Having this in mind its possible to retrieve all
  the classes in the model as a list.
  Then it is possible to go in depth in each one of this
  class elements retrieving attributes, stereotypes, operations
  and relations to other classes.
  Then again for each class sub-element it is possible
  to retrieve their sub-elements, so on and so forth
  until leaf elements are reached.

  \item{\textbf{Applying the templates.}}
  The OAW Xpand2 templates are defined as
  dynamically replacing text.
  This is done by defining routines that
  fetch the required attributes from the parsed
  model and
  replace it in a text template.
  It is possible to define multiple output text files
  as well as a single output file
  in a single template file.
  This allows the generator to generate one file
  per class, and
  a single file for common class interfaces (IDLs).
  The text which is replaced in the template
  is embraced within the special characters
  \textbf{\flqq~~\frqq} (called guillemets).
  The replacement text may be a call
  to a function in Xtend or even an external Java
  call in case very complex substitutions are needed.
  This flexibility allows the template to be very small,
  but generate substantial files, or even multiple files,
  keeping the complexity of the template as low as possible.
\end{itemize}

\subsection{Design}
\label{sec:design}
The integrated design of the generator may be seen in figure~\ref{fg:design}.
The overall design is based on translating an UML model in XMI format
into an ACS Component.
This is done using:
\begin{enumerate}
    \item\textbf{OAW Workflow definition:}
    This defines the steps to be taken to generate the ACS component.
    In the case of this specific generator
    the steps defined in the workflow are the ones
    described previously in table~\ref{tb:oawtasks}
    \item\textbf{Template definition:}
    Each one of the defined tasks requires a separate template that will
    translate into a file. Each template is domain specific, so there must be one for each transformation
    that must be done (i.e., Java files, CDB files, etc).
    The input information for these transformations
    is the UML model that has been previously parsed by OAW.
    \item\textbf{Extension definition:} These are meant to be helpers to the \textbf{Template definitions}.
    They gather common transformation that are shared by all the domains. In the specific case
    of this generator they gather functions to detect if a class should be generated or not,
    and in what way it should, if it is an IDL
    \lstinline[language=IDL]!Struct!,
    \lstinline[language=IDL]!Component!,
    or \lstinline[language=IDL]!CharacteristicComponent!.
\end{enumerate}

\begin{figure}
  \begin{center}
  \includegraphics[width=\textwidth]{images/design}
  \end{center}
  \caption{Generator Design}
  \label{fg:design}
\end{figure}

All generated files are output to a directory defined in the OAW workflow file.
This directory is created so as to mimic an ACS template.
For this reason,
once the files are generated its possible to compile and install right away.

\subsubsection{Challenges Encountered}
This section describes the specific problem and challenges that were encountered
when developing the ACS code generator.
\begin{itemize}
  \item{\textbf{The templates get cluttered.}}
  The templates get cluttered very fast with tasks
  and
  statements that must be repeated.
  To solve this, the repetitive task or
  similar tasks were transferred to an
  extension file.
  This extension file is written in a
  functional language that OAW provides.
  Extension may be included in any template
  and help to keep the template simple.
  \item{\textbf{Mapping of types from UML to code is not direct.}}
  UML defines native type with caps for the first letter.
  For example,
  IDL does not support the type \lstinline[language=Java]!int!,
  and JacOrb translates a
  \lstinline[language=IDL]!long! into an \lstinline[language=Java]!int!,
  this created many type
  mismatches. To solve this problem
  special methods where implemented in the extension file.
  The methods translate an UML type into an IDL or Java type
  as required.
  \item{\textbf{Return types and return values for the Java code are not correct.}}
  This problem is similar to the one described in the previous item.
  The issue is that the Java code must return the type and value defined in IDL,
  which led to mismatches. The solution was to implement a method that gets the proper
  return type and value for the Java code. This method is also part of the
  extensions file.
  \item{\textbf{Definition of IDL structs and enumerated types lead to type redefinitions.}}
  To address this issue, OAW tasks \lstinline[language=sh]!genidl! (described in table~\ref{tb:oawtasks})
  was internally split into two steps.
  \begin{enumerate}
    \item Generate IDL structs and enumerated types into a common file.
    \item Generate other IDL files and include the common file.
  \end{enumerate}
  This also provides a more granular control
  when importing this IDL files from client code
  or external software module.
  \item{\textbf{Generation does not guarantee the order of IDL entities}}.
  This problem appears when using more complex models. If the XMI description
  is not ordered in a dependency sense, then the generated IDL code
  does not build since it uses structures that have not been defined yet.
  Dependencies within classes may be solved by forward declaration, but
  dependencies within IDL structs may not.
  Therefore the XMI file must be dependency ordered\footnote{%
  An experimental ordering function is being tested to circumvent this issue.
  This function should be ready for the release of the second prototype.}.
\end{itemize}


%%% Local Variables:
%%% mode: latex
%%% TeX-master: "../thesis"
%%% End:
